\documentclass[12pt]{book}
 
\usepackage[margin=1in]{geometry}
\usepackage[utf8]{inputenc}
\usepackage{verbatim}
\usepackage{pgfplots}
    \pgfplotsset{compat=1.12,}
\usepackage{enumitem}
\usepackage{amsmath,amsfonts,amsthm,amssymb,graphicx,mathtools,hyperref}
\usepackage{wrapfig}
\usepackage{pdfpages}
\usepackage[export]{adjustbox}
\usepackage{tikz}
\renewcommand\qedsymbol{$\blacksquare$}
\usetikzlibrary{positioning}
\rerenewcommand{\A}{\mathbb{A}}
\renewcommand{\B}{\mathbb{B}}
\renewcommand{\C}{\mathbb{C}}
\renewcommand{\D}{\mathbb{D}}
\renewcommand{\E}{\mathbb{E}}
\renewcommand{\F}{\mathbb{F}}
\renewcommand{\G}{\mathbb{G}}
\renewcommand{\Hb}{\mathbb{H}} %
\renewcommand{\I}{\mathbb{I}}
\renewcommand{\J}{\mathbb{J}}
\renewcommand{\K}{\mathbb{K}}
\renewcommand{\Lb}{\mathbb{L}} %
\renewcommand{\M}{\mathbb{M}}
\renewcommand{\N}{\mathbb{N}}
\renewcommand{\Ob}{\mathbb{O}} %
\renewcommand{\Pb}{\mathbb{P}} % 
\renewcommand{\Q}{\mathbb{Q}}
\renewcommand{\R}{\mathbb{R}}
\renewcommand{\Sb}{\mathbb{S}} % 
\renewcommand{\T}{\mathbb{T}}
\renewcommand{\U}{\mathbb{U}}
\renewcommand{\V}{\mathbb{V}}
\renewcommand{\W}{\mathbb{W}}
\renewcommand{\X}{\mathbb{X}}
\renewcommand{\Y}{\mathbb{Y}}
\renewcommand{\Z}{\mathbb{Z}}

\newcommand{\Ac}{\mathcal{A}}
\newcommand{\Bc}{\mathcal{B}}
\newcommand{\Cc}{\mathcal{C}}
\newcommand{\Dc}{\mathcal{D}}
\newcommand{\Ec}{\mathcal{E}}
\newcommand{\Fc}{\mathcal{F}}
\newcommand{\Gc}{\mathcal{G}}
\newcommand{\Hc}{\mathcal{H}} %
\newcommand{\Ic}{\mathcal{I}}
\newcommand{\Jc}{\mathcal{J}}
\newcommand{\Kc}{\mathcal{K}}
\newcommand{\Lc}{\mathcal{L}} %
\newcommand{\Mc}{\mathcal{M}}
\newcommand{\Nc}{\mathcal{N}}
\newcommand{\Oc}{\mathcal{O}} %
\newcommand{\Pc}{\mathcal{P}} % 
\newcommand{\Qc}{\mathcal{Q}}
\newcommand{\Rc}{\mathcal{R}}
\newcommand{\Sc}{\mathcal{S}} % 
\newcommand{\Tc}{\mathcal{T}}
\newcommand{\Uc}{\mathcal{U}}
\newcommand{\Vc}{\mathcal{V}}
\newcommand{\Wc}{\mathcal{W}}
\newcommand{\Xc}{\mathcal{X}}
\newcommand{\Yc}{\mathcal{Y}}
\newcommand{\Zc}{\mathcal{Z}}
\newcommand{\ita}[1]{\textit{#1}}
\newcommand{\com}[2]{#1\backslash#2}
\newcommand{\oneton}{\{1,2,3,...,n\}}
\newcommand\idea[1]{\begin{gather*}#1\end{gather*}}
\newcommand\ef{\ita{f} }
\newcommand\eff{\ita{f}}
\newcommand\proofs[1]{\begin{proof}#1\end{proof}}
\newcommand\inv[1]{#1^{-1}}
\newcommand\setb[1]{\{#1\}}
\newcommand\en{\ita{n }}
\newcommand{\vbrack}[1]{\langle #1\rangle}
\DeclarePairedDelimiter\floor{\lfloor}{\rfloor}
\DeclareMathOperator{\ord}{ord}
\DeclareMathOperator{\cl}{cl}

\theoremstyle{plain}
\newtheorem{theorem}{Theorem}[section]
\newtheorem{lemma}[theorem]{Lemma}
\newtheorem{proposition}[theorem]{Proposition}
\newtheorem*{corollary}{Corollary}

\theoremstyle{plain} % just in case the style had changed
\newcommand{\thistheoremname}{}
\newtheorem*{genericthm}{\thistheoremname}
\newenvironment{namedtheorem}[1]
  {\renewcommand{\thistheoremname}{#1}%
   \begin{genericthm}}
  {\end{genericthm}}

\theoremstyle{definition}
\newtheorem*{definition}{Definition}
\newtheorem{conjecture}{Conjecture}
\newtheorem{example}{Example}[chapter]
\newtheorem*{HW}{Homework}

\theoremstyle{remark}
\newtheorem*{remark}{Remark}
\newtheorem*{claim}{Claim}
\newtheorem*{note}{Note}
\newtheorem*{problem}{Problem}
\newtheorem*{solution}{Solution}

\renewcommand*{\proofname}{Proof}


 
\title{Examen 2}
\author{Alec Zabel-Mena\\ 801-16-9720 \\
alec.zabel@upr.edu}
\date{\today}

\begin{document}

\maketitle

%\includepdf[pages=-]{}

\begin{enumerate}
    \item[(5.2)] 
        \begin{problem}
            Prove by induction on $m$ that  $m^3 \leq 2^m$.		
        \end{problem} 
        \begin{solution}
            For $m=10$, we have  $10^3=1000$ and  $2^^{10}=1024$, and $1000 \geq 1024$. Now
            suppose for  $m \geq 10$, that  $m^3 \geq 2^m$. Then
            $(m+1)^3=(m+1)(m+1)(m+1)(m^2+2m+1)(m+1)=m^3+(3m^2+3m+1) \geq m^3+m^3 \geq 2^m+2^m=2 \cdot
            2^m=2^{m+1}$.	
        \end{solution}

    \item[(5.3)] 
        \begin{problem}
            Prove by induction on $n$, for all positive integers n, $n \geq 1$.
        \end{problem}
        \begin{solution}
            For $n=1$,  $1 \geq 1$ is true. Now suppose that for all  $n \in \Z^+$ that  $n
            \geq 1$. By the axioms of Peano, we have that the successor of $n$ is  $s(n)=n+1>n\geq 1$,
            hence $n+1 \geq 1$.		
        \end{solution}

    \item [(8.2)] 
        \begin{problem}
            Define functions $f:\R \rightarrow \R$ and  $g:\R \rightarrow \R$ by $f(x)=x^3$ and
            $g(x)=1-x$. Find the functions:
                \begin{enumerate}[label=(\arabic*)]
                    \item $f \circ f$.

                    \item $f \circ g$.

                    \item  $g \circ f$.

                    \item  $g \circ g$.
                \end{enumerate}

                List the elements of the set $\{x \in \R:f \circ g(x)=g \circ f(x)\}$.
        \end{problem} 
        \begin{solution}
            Let $f,g:\R \rightarrow \R$ be defined by $f:x \rightarrow x^3$ and  $g:x
            \rightarrow 1-x$. Then  $f \circ f=f^2:\R \rightarrow \R$ is defined by  $f^2:x \rightarrow
            x^2 \rightarrow (x^3)^3=x^9$. $f \circ g:\R \rightarrow \R$ is defined by  $f \circ g:x
            \rightarrow 1-x \rightarrow (1-x)^3=(1-x)(1-x)(1-x)=(1-2x+x^2)(1-x)=x^3-x^2$. 
            $g \circ f:\R \rightarrow \R$ is defined by $g \circ f:x 
            \rightarrow x^3 \rightarrow 1-x^3$, and finally  $g \circ g=g^2:\R \rightarrow \R$ 
            is defined by  $g^2:x \rightarrow 1-x \rightarrow 1-(1-x)=x$.

            Now consider $f \circ g$ and  $g \circ f$, when is  $f \circ g=g \circ f$? Consider
            $x^3-x^2=1-x^3$. Then $x^2+1=0$, which has no solutions in $\R$. So $f \circ g \neq g
            \circ f$ for all  $x \in \R$.
        \end{solution}
        \begin{remark} 
            The second part of this problem may be misconstrued, as in my copy of the book it says
            to list all the elements of the set $\{x \in \R:fg(x)=gf(x)\}$; which may mean something
            different than function composition for some authors. However, it is most likely that
            composition is implied by that notation since in general $g \circ f \neq f \circ g$,
            moreover it is a popular algebraic notation to express composition by $fg$ instead of
            $f \circ g$.
        \end{remark}

    \item[(8.3)]\begin{problem}
                Find the functions $f_i:\R \rightarrow \R$ with the images as follows:
                    \begin{enumerate}[label=(\arabic*)]
                        \item $ f_1(\R)=\R$.
                
                        \item $f_2(\R)=\R^+$.

                        \item $f_3(\R)=\com{\R}{\Z}$.

                        \item $f_4(\R)=\Z$.
                    \end{enumerate}
            \end{problem}
            \begin{solution}
                \begin{enumerate}[label=(\arabic*)]
                    \item Take $f_1:\R \rightarrow \R$ by $ f_1:x \rightarrow x^3$. Then $ f_1(\R)=\R$.		

                    \item Take $ f_2:\R \rightarrow \R$ by $f_2:x \rightarrow \begin{cases}
                                                    x \text{, } 0 < x < 1 \\
                                                    1 \text{, } x \geq 1
                                              \end{cases}$.
                        Then $ f_2(\R)=\R^+$.

                    \item Take $ f_3:\R \rightarrow \R$ by $f(x)=x-\floor{x}$; where $\floor{x}$ is the
                      greatest interger less or equal to $x$. Then  $ f_3(\R)=\com{\R}{\Z}$.

                    \item Take $ f_4:\R \rightarrow \R$ by $ f_4:x \rightarrow \floor{x}$. Then $ f_4(\R)=\Z$.
                \end{enumerate}
            \end{solution}

\end{enumerate}

\end{document}
