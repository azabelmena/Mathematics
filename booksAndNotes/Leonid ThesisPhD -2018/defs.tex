% defs.tex
%==========================================================================

% INICIO- paquetes del articulo-Leonid


%\usepackage[inner=1in,outer=1in,bottom=1in,top=1in]{geometry}
%\usepackage[sort&compress]{natbib}
%\usepackage[titletoc,toc,title]{appendix}
%%%%%%%%%%%%%%%%%%%%%%%%%%%%%%%%%%%%%%%%%%%%%%%%%%%%%%%%%%%%%
% New definitions	
%%%%%%%%%%%%%%%%%%%%%%%%%%%%%%%%%%%%%%%%%%%%%%%%%%%%%%%%%%%%%


% AMS article
%\documentclass{amsart}

%\usepackage{graphicx}
%\usepackage{amsfonts}
%\usepackage{amssymb}
%\usepackage{amsmath}
%\usepackage{color}
%\usepackage[inner=1in,outer=1in,bottom=1in,top=1in]{geometry}
%\usepackage[sort&compress]{natbib}
%\textwidth=6in \textheight=9in \hoffset=-0.375in \voffset=-0.75in
%% Style definitions and "newtheorems"
%\numberwithin{equation}{section}
%\newtheorem{theorem}{Theorem}[section]
%\newtheorem{corollary}[theorem]{Corollary}
%\newtheorem{proposition}[theorem]{Proposition}
%\newtheorem{conjecture}[theorem]{Conjecture}
%\newtheorem{lemma}[theorem]{Lemma}
%\newtheorem{theorem}{Theorem}[section]
%\theoremstyle{definition}
%\newtheorem{definition}[theorem]{Definition}
%\newtheorem{observation}[theorem]{Observation}
%\newtheorem{example}[theorem]{Example}
%\newtheorem{xca}[theorem]{Exercise}
%\theoremstyle{remark}
%\newtheorem{remark}[theorem]{\bf\em Remark}

%\usepackage[titletoc,toc,title]{appendix}
%%%%%%%%%%%%%%%%%%%%%%%%%%%%%%%%%%%%%%%%%%%%%%%%%%%%%%%%%%%%%
% New definitions	
%%%%%%%%%%%%%%%%%%%%%%%%%%%%%%%%%%%%%%%%%%%%%%%%%%%%%%%%%%%%%
%\newcommand{\X}{\mathbf{X}}
\newcommand{\xx}{\mathbf{x}}
\newcommand{\ff}{{\mathbb{ F\!}}}
\newcommand{\leg}[2]{\left(\frac{#1}{#2}\right)}
%\newcommand{\Tr}{\mathrm{Tr}}
%\newcommand{\tr}{{\operatorname{Tr}}}
\newcommand{\Sym}{\mathrm{Sym}}
\def\XX{{\bf X}}
\def\aa{{\bf a}}
\def\+{{\oplus}}

%\makeatletter
%\def\Ddots{\mathinner{\mkern1mu\raise\p@
%\vbox{\kern7\p@\hbox{.}}\mkern2mu
%\raise4\p@\hbox{.}\mkern2mu\raise7\p@\hbox{.}\mkern1mu}}
%\makeatother

\DeclareMathOperator{\lcm}{lcm}
\DeclareMathOperator{\ord}{ord}
\DeclareMathOperator{\CIRC}{circ}

% FIN-  paquetes del articulo-Leonid



%===================================================

\theoremstyle{plain}
\newtheorem{definition}[subsection]{Definition}    % This defines the Definition enviroment
\newtheorem{conjecture}[subsection]{Conjecture}
\newtheorem{openquestion}[subsection]{Open question}
\newtheorem{question}[subsection]{Question}																									 % Ver capitulo 5.
\newtheorem{remark}[subsection]{Remark}
\newtheorem{example}[subsection]{Example}										 % This is example
\newtheorem{proposition}[subsection]{Proposition}																									 % Ver capitulo 5.
\newtheorem{lemma}[subsection]{Lemma}
\newtheorem{theorem}[subsection]{Theorem}											 % This is the theorem formulation heading
\newtheorem{corollary}[subsection]{Corollary}																							 % Ver capitulo 5.

\newtheorem*{proofa}{Proof}

\hypersetup{urlcolor=blue}			 % Especifica el azul para los hypervinculos. (pags web)

\newcommand{\qfd}{\hfill $\fbox{}$\vspace{2mm}}

\newcommand{\fn}[1]{\texttt{#1}}						% Estas dos lineas son utiles para el capitulo 4.
\newcommand{\cn}[1]{\texttt{\char92 #1}}

\newcommand{\nt}{\triangledown_2}
\newcommand{\dt}{\triangle_2}



\newcommand{\re}{\mathbb{R}}
\newcommand{\CCC}{{\mathfrak C}}
\newcommand{\bR}{\overline{\R}}
\newcommand{\N}{{\mathbb N}}
\newcommand{\Z}{{\mathbb Z}}

\newcommand{\Q}{{\mathbb Q}}
\newcommand{\X}{{\mathbb X}}
\newcommand{\T}{{\mathbb T}}
\newcommand{\D}{{\mathbb D}}
\newcommand{\cB}{{\mathcal B}}
%\newcommand{\C}{{\mathbb C}}
\newcommand{\cD}{{\mathcal D}}
\newcommand{\cN}{\mathcal{N}}
\newcommand{\cC}{{\mathcal C}}
\newcommand{\cE}{{\mathcal E}}
\newcommand{\cF}{{\mathcal F}}
\newcommand{\cL}{{\mathbb L}}
\newcommand{\bF}{\bar{\mathcal F}}
\newcommand{\cT}{{\mathbf T}}

\newcommand{\sgn}{{\operatorname{sgn}}}
\newcommand{\tr}{{\operatorname{Tr}}}
\newcommand{\nm}{\operatorname{\mathsf N}}
\newcommand{\inter}{\operatorname{Int}}

\newcommand{\cA}{\mathcal{A}}
\newcommand{\cS}{{\mathcal S}}
\newcommand{\cU}{{\mathcal U}}
\newcommand{\cV}{{\mathcal V}}
\newcommand{\cH}{{\mathcal H}}
\newcommand{\cK}{{\mathcal K}}
\newcommand{\cM}{{\mathcal M}}
\newcommand{\cR}{{\mathcal R}}
\newcommand{\loc}{\rm{loc}}


\newcommand{\Capw}{\operatorname{Cap}}
\newcommand{\Capr}{\operatorname{Cap}_{\Om}}
\newcommand{\dom}{\operatorname{dom}}



\newcommand{\NN}{\mathbb{N}}
\newcommand{\IN}{\mathbb{N}}
\newcommand{\CC}{\mathbb{C}}
\newcommand{\ZZ}{\mathds{Z}}
\newcommand{\K}{\mathbb{K}}
\newcommand{\Dw}{\mathbb{D}}
\newcommand{\Qw}{\mathbb{Q}}
\newcommand{\Om}{\Omega}
\newcommand{\De}{\Delta}
\newcommand{\ep}{\epsilon}
\newcommand{\var}{\varepsilon}
\newcommand{\si}{\sigma}
\newcommand{\lla}{\lambda}
\newcommand{\om}{\omega}
\newcommand{\al}{\alpha}
\newcommand{\de}{\delta}
\newcommand{\ga}{\gamma}
\newcommand{\Ga}{\Gamma}
\newcommand{\bOm}{\overline{\Om}}
\newcommand{\vep}{\varepsilon}
\newcommand{\pOm}{\partial\Omega}
\newcommand{\sgnw}{\operatorname{sgn}}
\newcommand{\op}{\operatorname{Re}}

