% GeneralAudienceAbstract.tex


This is the general Audience Abstract.

Use the file: \fn{GeneralAudienceAbstract.tex}

%Density matrix theory is applied to the CO/Cu(001)
%adsorbate/substrate system to evaluate the nonlinear photodesorption yield of CO versus pulse fluence through model calculations. The dynamics of molecular photodesorption from a metal surface is described by the nonlinear optical response resulting from the interaction of a femtosecond pulsed laser with a metal surface. Our formalism uses the Liouville-von Neumann equation, with an effective hamiltonian which includes the effects of energy dissipation into the metal. The nonlinear response of the substrate to femtosecond excitation is taken into account by solving modified optical Bloch equations with relaxation terms to account for the effects of energy dissipation. Our previous density matrix treatment has been extended to include several quantum states, and to treat rotational motions of the desorbing molecule. Results for intense lasers including chirping at pulse wavelengths around 620 nm have also been obtained.
