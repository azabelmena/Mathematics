%----------------------------------------------------------------------------------------
%	SECTION 1.1
%----------------------------------------------------------------------------------------

\section{Inner Product Spaces.}
\label{section1}

\begin{definition}
    We define a vector space $V$ over $\C$ to be an  \textbf{inner product space} if there
    exists a binary operation $\vbrack{,}:V \times V \rightarrow \C $ such that for all  $v,u,w \in V$
    and  $\alpha, beta \in \C$:
        \begin{enumerate}
            \item[(1)] $\vbrack{u,v}=\bar{\vbrack{v,u}}$.

            \item[(2)] $\vbrack{u,u} \geq 0$ and $\vbrack{u,u} = 0$ if and only if  $u=0$.

            \item[(3)] $\vbrack{\alpha u+\beta v, w}=\alpha\vbrack{u,w}+\beta\vbrack{v,w}$.

        \end{enumerate}
\end{definition}

\begin{example}
    \begin{enumerate}		
        \item[(1)] In $\C^n$, let $u=(\alpha_1, \dots, \alpha_n)$ and $v=(\beta_1, \dots, \beta_n)$ and
            define $\vbrack{u,v}=\sum_{i=1}^n{\alpha_i\bar{\beta_i}}$. Notice that 
            $\sum{\alpha_i\bar{\beta_i}}=\sum_{i=1}^n{\bar{\beta_i}\alpha_i}=\bar{\sum{\bar{\alpha_i}\beta_i}}$;
            so $\vbrack{u,v}=\bar{\vbrack{v,u}}$. We also have that $\vbrack{u,u} \geq 0$ and is
            $0$ only when  $u=0$. Moreover, if  $w=(\gamma_1, \dots, \gamma_i)$ and
            $\alpha,\beta \in \C$, then  $\vbrack{\alpha u+\beta
            v,w}=\sum{(\alpha\alpha_i+\beta\beta_i)\bar{\gamma_i}}=\alpha\sum{\alpha_i}\bar{\gamma_i}+
            \beta\sum{\beta_i}\bar{\gamma_i}=\alpha\vbrack{u,w}+\beta\vbrack{v,w}$. So
            $\vbrack{,}$ defines an inner product over $\C^n$.

        \item[(2)] Let $\C^{[0,1]}$ be the set of all complex valued functions continous on the
            domain $[0,1]$. If $f,g \in \C^{[0,1]}$, define
            $\vbrack{f,g}=\int_{0}^1{f(t)\bar{g(t)}}\dd{t}$. Then $\vbrack{,}$ defines an inner
            product over $\C^{[0,1]}$. Let $f,g,h \in \C^{[0,1]}$ and $\alpha,\beta \in \C$. We have
            then that
            $\vbrack{f,g}=\int{f\bar{g}}=\int{\bar{\bar{f}g}}=\bar{\int{\bar{f}g}}=\bar{\vbrack{g,f}}$.
        Moreover, $\int_{0}^1{f\bar{f}}\dd{t} \geq 0$; now  $\vbrack{f,f}=0$ if $f=0$. Now if
            $\int{f\bar{f} \dd{t}=0}$, letting  $f(t)=x(t)+iy(t)$, by the product of conjugates, and
            the sum rule, $x(t)=y(t)=0$, i.e. $f=0$. Again, by the rules of complex integras,
            $\vbrack{\alpha f+\beta g,h}=\int{(\alpha f+\beta
            g)\bar{h}}=\alpha\int{f\bar{h}}+\beta\int{g\barh{h}}$.
    \end{enumerate}
\end{example} 

\begin{definition}
    Let $V$ be an inner product space over  $\C$. The  \textbf{norm} of $v \in V$ is the map
    $||\cdot||:V \rightarrow \R$ by  $||v||=\sqrt{\vbrack{v,v}}$.
\end{definition}

\begin{lemma}\label{1.4.1}
    If $V$ is an inner product space, with  $u,v \in V$ and  $\alpha,\beta \in \C$, then
    $\vbrack{\alpha u+\beta v, \alpha u+\beta
    v}=\alpha\bar{\alpha}\vbrack{u,u}+\alpha\bar{\beta}\vbrack{u,v}+\bar{\alpha}\beta\vbrack{v,u}+\beta\bar{\beta}\vbrack{v,v}$.
\end{lemma}
\begin{proof}
    Take $(3)$ on the inner product $\vbrack{\alpha u+\beta v, \alpha u+\beta v}$ to get: 
    $\vbrack{\alpha u+\beta v, \alpha u+\beta v}=\alpha\vbrack{u,\alpha u+\beta v}+\beta\vbrack{v,
    \alpha u+\beta v}=\alpha\bar{\vbrack{\alpha u+\beta v,u}}+\beta\bar{\vbrack{\alpha u+\beta
v,v}}=\alpha\bar{\alpha}\vbrack{u,u}+\alpha\bar{\beta}\vbrack{u,v}+\bar{\alpha}\beta\vbrack{v,u}+\beta\bar{\beta}\vbrack{v,v}$.
\end{proof}
\begin{corollary}
    $||\alpha u||=|\alpha|||u||.$
\end{corollary}
\begin{proof}
    We have $||\alpha u||^2=\vbrack{\alpha u, \alpha u}=\alpha\bar{\alpha}\vbrack{u,u}$. Since
    $\alpha\bar{\alpha}=|\alpha|^2$ we have $||\alpha u||=|\alpha|^2||u||^2$ which gives us the
    result.
\end{proof}
