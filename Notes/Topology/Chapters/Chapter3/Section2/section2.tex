%----------------------------------------------------------------------------------------
%	SECTION 1.1
%----------------------------------------------------------------------------------------

\section{Connected Spaces of The Real Line.}

\begin{definition}
    We call a simply ordered set $L$ with  $|L|>1$ a  \textbf{ordered contunuum} if:
        \begin{enumerate}[label=(\arabic*)]
            \item $L$ has the least upperbound property.

            \item If $x<y$, then there exists a  $z$ such that  $x<z<y$.
        \end{enumerate}
\end{definition}

\begin{theorem}\label{3.2.1}
    If $L$ is a linear continuum in the order topology, then  $L$ is connected, and so are the open
    sets of  $L$  (the intervals and rays in $L$).
\end{theorem}
\begin{proof}
    We show that convex sets are connected. Let $Y=A \cup B$ be a seperation, and choose  $a \in A$,
     $b \in B$ with  $a<b$. We have that the interval of points in  $L$,  $[a,b] \subseteq Y$; and
     we also have that $[a,b] \subseteq A_0 \cup B_0$ with $ A_0=A \cap [a,b]$ and $ B_0=B \cap
     [a,b]$. Now $ A_0,B_0 \neq \emptyset$, so $[a,b]=A_0 \cup B_0$ is a seperation of $[a,b]$. Now
     let $c=\sup{A_0}$. Suppose first that $c \in B_0$, then c \neq a, so either $c=b$ or  $a<c<b$.
     Since  $ B_0$ is open in $[a,b]$ as a subspace of $Y$, there is some interval  $(d,c] \subseteq
     B_0$.

     If $c=b$, then  $d<c$ is an upperbound of  $ A_0$, which contradicts that $c$ is the least
     upperbound. Now suppose that $c<b$. We have that since  $c,b \in B_0$, $(c,b] \cap A_0=
     \emptyset$, then $(b,d] \cap A_0=(d,c] \cap (c,b] \cap A_0 = \emptyset$, and again we have
     $d<c$ which gives us the contradiction. So  $c \notin B$. By similar reasoning  $c \notin A_0$.
\end{proof}
\begin{corollary}
    $\R$ is connected and so are the intervals and rays of  $\R$.
\end{corollary}
\begin{proof}
    $\R$ is a linear continuum.
\end{proof}

\begin{theorem}[The Intermediate Value Theorem]\label{3.2.2}
    Let $f:X \rightarrow Y$ be continuous with  $X$ connected, and  $Y$ an ordered set under the
    order topology. If  $a,b \in X$, and if  $r \in Y$ such that  $f(a)<r<f(b)$ or $f(b)<r<f(a)$,
    then there exists a $c \in X$ for which  $f(c)=r$.
\end{theorem}
\begin{proof}
    Let $r \in Y$ such that  $f(a)<r<f(b)$, without loss of generality. We have that $A=f(X) \cap
    (-\infty)$ and $B=f(X) \cap (r ,\infty)$ are disjoint, nonempty sets open if $f(X)$ as a
    subspace of $Y$. Now suppose there is no  $c \in X$ for which  $f(c)=r$, then $f(X)=A \cup B$ is
    a seperation of $f(X)$, which contradicts theorem \ref{3.1.6}.
\end{proof}

\begin{example}
\end{example} 
